\chapter*{Summary}
\addcontentsline{toc}{chapter}{Summary}

City of Rotterdam oversees an extensive network of approximately 2000 groundwater monitoring wells, which are crucial for monitoring the groundwater dynamics across both urban and rural settings. The network is essential for addressing issues on foundations, subsidence, and (ground)water nuisance in basements throughout the municipal area, but is not intended to be used on a household level. Despite the distribution of the wells across the neighborhoods, the uniformity of the monitoring wells differs across them. City of Rotterdam only investigates the coverage of the neighborhoods when a well is expired: the necessity of well replacement is only considered. However, a reduction or extension optimization approach is not considered. This resulted in a stagnant number of monitoring wells over the past years, prompting questions regarding the level of optimization of the network in capturing comprehensive groundwater level data. Groundwater plays a pivotal role in the urban hydrological cycle. Consequently, the research aims to examine the level of optimization, questioning the density of the network and the location of the monitoring wells throughout the case study areas Rozenburg and Heijplaat within City of Rotterdam.\\
\\
The methodology for evaluating Rotterdam’s groundwater monitoring network (GWMN) employs a two-part design: descriptive and quasi-experimental. At first, the approach characterizes the current state of the GWMN by evaluating a hydrogeological system description and by visualizing the well distribution and groundwater variables without manipulating the existing groundwater level data. Subsequently, the network is manipulated by simulating groundwater level data based on the Python package \textit{Pastas }that is used to incorporate time series modeling of groundwater level data. QR factorization optimizes the network by identifying the informational relevance of the monitoring wells, based on a hierarchical list. The approach enables a data-driven approach to enhance monitoring network efficiency across the case study areas Rozenburg and Heijplaat.\\
\\
The research study critically examines the optimization of Rotterdam’s GWMN through a time-focused analysis and reduction-optimization model. The observed data from 2010 to 2024 highlights the importance of choosing an appropriate time frame for capturing groundwater dynamics accurately. Utilizing \textit{Pastas}, the study bridges data gaps and employs QR factorization for strategic well reduction. The study finds the statement that an optimal GWMN does not necessarily rely on a high data quantity, but on strategic placement of monitoring wells and data quality. A 25 percent reduction emerged as a suitable approach, enhancing the network’s efficiency without compromising data integrity. The methodology offers a model for optimizing GWMNs, applicable across Rotterdam’s diverse neighborhoods, aligning with the Pareto Optimum principle for maximizing informational relevance and minimizing expenses. Overall, the study underscores the balance between monitoring network density and data quality, revealing an approach to a more optimized GWMN, but also addresses implications that extend beyond the municipal area and aims towards future-oriented urban water management.

\chapter*{Samenvatting}
\addcontentsline{toc}{chapter}{Samenvatting}

Gemeente Rotterdam houdt toezicht op een uitgebreid netwerk van ongeveer 2000 peilbuizen die cruciaal zijn voor het monitoren van de grondwaterdynamiek in zowel stedelijk als landelijk gebied. Het netwerk is essentieel voor de aanpak van vraagstukken met betrekking tot funderingen, verzakkingen en (grond) wateroverlast in kelders in het gehele gemeentegebied, maar is niet bedoeld voor gebruik op het niveau van huishoudens. Ondanks de verdeling van de peilbuizen over de wijken verschilt de uniformiteit per wijk. Gemeente Rotterdam onderzoekt alleen de dekking van de buurten wanneer een peilbuis is verlopen: er wordt gekeken naar de noodzaak van vervanging. Mogelijkheden voor reductie of uitbreiding van de peilbuizen wordt niet onderzocht. Dit resulteerde de afgelopen jaren in een stagnerend aantal peilbuizen, wat vragen opriep over de mate van optimalisatie van het netwerk bij het vastleggen van uitgebreide grondwaterdata. Grondwater speelt een cruciale rol in de stedelijke hydrologische cyclus. Het onderzoek heeft daarom als doel het niveau van optimalisatie te onderzoeken, waarbij de dichtheid van het netwerk en de locatie van de peilbuizen in de onderzoeksgebieden Rozenburg en Heijplaat binnen de Gemeente Rotterdam onderzocht worden. \\
\\
De methodologie voor het evalueren van het Rotterdamse achtergrondmeetnet maakt gebruik van een tweedelig ontwerp: beschrijvend en quasi-experimenteel. Als eerste wordt de huidige toestand van het netwerk behandeld, waar een hydrogeologische systeembeschrijving en verdeling van de peilbuizen beschreven en gevisualiseerd wordt zonder de bestaande grondwaterdata te manipuleren. In de tweede stap wordt het netwerk gemanipuleerd door grondwaterdata te simuleren op basis van het Python pakket \textit{Pastas}, dat gebruikt wordt voor tijdreeksanalyse van grondwaterdata. QR factorisatie gebruikt een optimalisatie methode door de informatie relevantie van de peilbuizen te identificeren op basis van een hiërarchische lijst. De aanpak maakt gebruik van een data gestuurde aanpak, mogelijk om de efficiëntie van het meetnet in de onderzoeksgebieden Rozenburg en Heijplaat te analyseren en te verbeteren. \\
\\
Het onderzoek analyseert kritisch de mate van optimalisatie van het Rotterdamse achtergrondmeetnet door middel van tijdreeksanalyse en een reductie-optimalisatiemodel. De geobserveerde data van 2010-2024 benadrukt het belang van het kiezen van een geschikte onderzoeksperiode voor het nauwkeurig vastleggen van de grondwaterdynamiek. Met behulp van \textit{Pastas} overbrugt het onderzoek datalacunes, waarna in de methode gebruik gemaakt wordt van QR factorisatie voor een strategische peilbuisreductie. De studie komt tot de conclusie dat een optimaal GWMN niet noodzakelijkerwijs afhankelijk is van een grote kwantiteit data, maar van datakwaliteit en strategische plaatsing van peilbuizen. Een reductiepercentage van 25 procent bleek een geschikte aanpak, waardoor de efficientie van het netwerk verbeterd kan worden zonder de data integriteit in gevaar te brengen. De methodologie biedt een model voor het optimaliseren van het GWMN, toepasbaar binnen de Gemeente Rotterdam en in lijn met het Pareto Optimum principe voor het maximaliseren van de informatie relevantie en het minimaliseren van de kosten van het GWMN. De studie markeert de balans tussen het monitoren van de netwerkdichtheid en de gegevenskwaliteit, waardoor een aanpak voor een meer geoptimaliseerd GWMN worden onthuld, maar ook de implicaties worden behandeld die verder reiken dan het gemeentelijk gebied en gericht zijn op toekomstgericht stedelijk waterbeheer.


