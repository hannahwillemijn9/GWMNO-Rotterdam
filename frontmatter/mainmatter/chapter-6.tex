\chapter{Discussion}
\label{chapter:discussion}

\emph{TO WRITE}


How does reducing the number of monitoring wells contribute to the development of reconstruction errors? (When the n of wells is high, is the rmse also high? If the n is low, is the rmse also low? Low rmse means a higher level of optimization). 
A clarification for a low RMSE when eliminating more monitoring wells from the study area are the following: 
- Selection of relevant monitoring wells. If the algorithm effectively eliminates the least informative monitoring wells identifies and removes, the monitoring wells that have the most relevant informative content remain in the area. These monitoring wells are the most precise for the reconstruction of the groundwater levels. The remaining monitoring wells gives a representation of the underlying dynamics of the groundwater levels in the area.
- Improved model performance. Through the removal of noise and redundant data, the model can be run more efficiently. Less variability in the dataset can lead to an improved generalization of the model, lowering the RMSE.
- Optimization of the data set. The process of removing monitoring wells can lead to optimization of the data set, where only the most significant and informative details are used for model training. 
- Prevent overestimating of the model. With a lower number of monitoring wells, the model prevents overestimating, because the model is being fitted to the training data. The forecasting accuracy of the model can be improved for the generation of new data. 
- Quality over quantity.  A higher amount of data available is not necessary better. A small dataset with a higher quality data can lead to a better model performance than a big dataset with lots of noise and irrelevant information. 









Manual measurements between 2014 and 2018 were executed by an external contracting company. The outliers in the manual measurements within this time period are probably the result of the work of  the contracting company. 

- Ranking 
The ranks are assigned a low number (essential well with high information content) to high number (most redundant well). Lower numbers mean a higher ranking concerning their information content and importance to reconstructing potentially removed redundant wells. 

%- Ranking map 
%Dark red = redundant/removed, minor loss in prediction accuracy 
%Dark blue = essential information about the system 
%- Important wells show flashiness, meaning high frequency, short-term changes 
%- Wells with high seasonality have less recharge through stream infiltration leading to short-term variations or flashiness. 

%- Python script is a decision-making tool.

- Remove monitoring wells with low information content and equip monitoring wells with a higher ranking with a higher time frequency or quality sensors for example. However, as the number of wells increases, the accuracy improves on average, hydrographs are reproduced more consistently, and short-term peaks are reproduced more accurately. 

%- Difference 1D and 2D methods: ranking based on 2D (spatially interpolated data) is different from ranking based on 1D (hydrograph data). This can be explained by the ranking reflecting the information content regarding the reconstruction with the lowest possible error. 

%1D: goal is to reconstruct the hydrographs of removed wells.  
%2D: goal is to reconstruct the interpolated surface. 

- 2D reduction and extension based on gridded GWL maps is to ensure that existing wells remain. A gradual cost function can be used where the weighting increases with distance to the infrastructure or similar. Weighting depends on the system, basis, and cost function and must be adjusted to a specific case. 

- Reconstruction of a large number of similar values is much easier for the model (around 400-500 wells), because there are more training patterns for each type of dynamics than with hydrographs alone. And the overall dynamics are reduced by interpolation with polishes the data spatially and temporally. 
- Maximum Absolute Error: is a good focus because the identification of areas with high errors, where the model cannot produce reliable reconstructions and additional wells would bring more information. 

QR Factorization
%Hydrographs of the reconstructed/removed monitoring wells are developed. Every hydrograph explains the statistical variables: MAE, KGE, NSE, R2 (rsq), RMSE, rBIAS.

%MAE = mean absolute error.
%> The smaller the MAE, the better the performance. 
%> 2D method appears to calculate smaller MAE's than 1D method.

%KGE = Kling-Gupta efficiency. 
%> Goodness of fit indicator for comparing simulations to observations. 
%> A KGE value of -0.41 and higher indicates an improved model. 
%- KGE = 1, perfect model correspondence. 
%Source: \href{https://hess.copernicus.org/preprints/hess-2019-327/hess-2019-327.pdf}{hess-2019-327.pdf (copernicus.org)} 

%NSE = Nash-Sutcliffe efficiency. 
%> Similar to KGE, KGE visualizes the shortcomings of the NSE. 
%- NSE = 1, perfect model correspondence. 
%Source: \href{https://hess.copernicus.org/articles/27/1827/2023/hess-27-1827-2023.pdf}{hess-27-1827-2023.pdf (copernicus.org)} 

%rBIAS = relative bias. 
%> Optimal rBIAS is 0. Positive values = underestimation, negative values = overestimation of the bias. 

%- Werkzaamheden bekijken en eventueel toevoegen aan de onderzoeksgebieden. 
 %- EVP in de tabel toevoegen: niet in de barplot. 