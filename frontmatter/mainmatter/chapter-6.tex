\chapter{                                                                                                       Discussion}
\label{chapter:discussion}
%\emph{The discussion first explains a brief summary of the results of the methodology, the results are further interpreted. The main question is: What do the results mean in the context of GWMNO and why do these results matter for the research study? Besides the interpretation and implications of the research study, the limitations also have to be addressed. It is of additional relevance to have the limitations of the research study clear.}\\
\\
\textbf{Observed data}\\
\\
The substantiation for the time period of the observed data, ranging from 2010-2024 to a more narrow scope of 2020-2024, could potentially affect the precision and reliability of the results of the stress model. A wider time frame may provide a better view of the groundwater dynamics and anomalies, referring to the outliers observed in the observed, manual measurements between 2014 and 2018. Nevertheless, an extended time period might introduce the variability that does not improve model performance. A shorter time frame, however, might not capture the entire range of hydrological events that are necessary to create a robust model. On the other hand, a longer time frame can positively influence the results, but they have to take into account the effects of external factors such as spatial planning of public spaces or inefficient sewage systems that could influence the variability of the data.    \\
\\

\textbf{Data preparation}\\
\\
Since a combination of dataloggers and a manual collection method is applied for many neighborhoods in the municipality, data gaps were identified. Time series modeling appeared to be an effective approach for bridging data gaps with dependable data. The Pastas time series modeling approach is based on simulation data of precipitation [m/day] and potential evaporation [m/day] data, in order to calculate recharge [m/day], of weather station 344 in Rotterdam, The Netherlands. Weather station 344 is the station that is commonly used to retrieve data for the municipality. However, as the crow flies, both Rozenburg and Heijplaat have a distance of maximum 15 kilometers from the weather station.

\textbf{QR factorization}\\
\\
The process of QR factorization and the additional reduction of monitoring wells introduces the relationship between the number of monitoring wells and the value of the metric RMSE. A general observation is made that a high level of optimization is characterized by a low RMSE, unlike assumptions that a low reduction rate corresponds to a high RMSE and so a high level of optimization when looking at data quantity. The RMSE could be closely related to the quality of the data and the selection process of the monitoring wells instead to data quantity. \\
\\
\textbf{Sensitivity analysis and performance metrics}\\
\\
It is necessary to question whether the RMSE and MAE are the only sensitive parameters that reflect the model accuracy. The RMSE describes a measure of the magnitude of errors and it remains unaffected by changes in the variability of the input data (the observed groundwater level data). The MAE, on the other hand, is influenced by the variability. Monitoring wells with a lower variability are likely to have less informational relevance to the network and are more likely to be eliminated. \\
\\
\textbf{Implications of reduction rates}\\
\\
Regarding the ideal network reduction rate, by reducing the network with more than 20\%, the remaining network is classified as impractical. Nevertheless, when the reduction rate reaches 25\%, monitoring wells with the lowest information content are identified and eliminated from the network. With a reduction rate of 25\%, the network is polished, but this did not influence the prediction accuracy. In the research study, selective elimination takes place based on a hierarchical list, which results in a lower loss of prediction accuracy of reconstruction accuracy of the eliminated monitoring wells. \\
\\
\textbf{Model robustness and variability}\\
\\
The RMSE remains unchanged when increasing or decreasing the variability of the input data. This might suggest that the model performance is consistent, the robustness to changes in variability is relatively high. Determining the robustness is necessary to examine whether the model is able to remain consistent when experiencing a variability of hydrological conditions and input scenarios. \\
\newline
Overall, the elimination of monitoring wells and the choice of study period require an equilibrium regarding data quality and quantity.  Both the RMSE and MAE are performance metrics that can evaluate model performance, because they ensure perspective in model sensitivity and performance. It has become clear that the RMSE is not sensitive to changes in variability input. At the same time, the direct influence of the MAE on the results marks the complexity of GWMNO. \\
\\
\textbf{Implementation of the model in the field}\\
\\
Before the research study started, the objective was set to create a generic model to be able to test the level of optimization for the entire municipal area of Rotterdam. Two case study areas were selected, Rozenburg and Heijplaat, to test and train the model with a lower number of monitoring wells. A query is, however, if the methodology and model would be robust enough to handle external factors, such as construction works, that would possibly have an influence on groundwater levels locally. The reduction-optimization approach is only being able to function if the network continues in its current state (March 2024).


