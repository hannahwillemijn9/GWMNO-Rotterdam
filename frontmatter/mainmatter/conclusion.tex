\chapter{Conclusion}
\label{chapter:conclusion}

The exploration of the groundwater monitoring network optimization within City of Rotterdam is the central point of this research study. The study addresses the following research question:\\
\\
\textit{"To what extent is the groundwater monitoring network of City of Rotterdam optimal?"}
\newline
\newline
Sub questions that were introduced in chapter \labelcref{chapter:introduction} \textit{"Introduction"} are answered.
\newline
\newline
1. What constitutes an optimal groundwater monitoring network, particularly in terms of the ideal reduction in monitoring wells, to maintain effective monitoring while maximizing efficiency? \\
\\
Theoretically, an optimal groundwater monitoring network is defined as a network that achieves the maximum possible information content at the lowest feasible expenses, which aligns with the Pareto Optimum principle. This means that the lowest number of monitoring wells ensures the highest quality of data through limiting construction measures regarding removal and reconstruction. An optimal GWMN with high quality data does not necessarily mean that high quantities data need to be available from the monitoring wells in the area. A network can be optimized by improving the density and measuring frequency of the monitoring wells. In practice, the placement of the monitoring wells should be optimal to ensure that they cover critical areas, vulnerable to the distance of a water body, construction works that could influence surface runoff and sewage system, the filter construction in relation to clay or sand. Next to that, the network should be able to retrieve high quality data rather than high quantities data. Every monitoring well should be equipped with certified sensors to ensure accurate data. Ultimately, a GWMN reaches its optimal state when it gathers crucial data through efficient placement of monitoring wells with data loggers, ensuring saving of expenses among others. The approach focuses on quality over quantity of data to maximize the network's impact and effectiveness in managing groundwater resources. For edification: Recommendations regarding including cost analysis in the model are described in the chapter \labelcref{chapter:recommendations}.\\
\\
2. How is the hydrogeological system characterized within the municipal boundaries of Rotterdam? \\
\\
\noindent
The hydrogeological system of City of Rotterdam can be addressed as a fairly complex heterogeneous formation. The deeper Holocene layer is characterized by formations that are marked by local gully erosion and cuts with layers that differ in structure, composition, density and permeability. Due to the influence of the North Sea and adjacent rivers in The Netherlands, there is a significant spatial variation in the Holocene layer. The Holocene layer is followed by a phreatic formation that is called the Anthropocene layer. This layer is permeable and the thickness varies between centimeters to 30 meters at the western side of the municipal area, reaching to the dunes of Hoek van Holland. In the urban areas, the top layer is strongly affected by anthropocentric influences. Like the Holocene layer, the Anthropocene layer varies in composition, but mainly consists of sand. 

Zooming into the case studies, in Rozenburg the Holocene formations consist of a 21 meter thick layer, which consists of sandy-clay, medium-fine sand, peat, and coarse sand. The presence of these sediments suggest a varied deposition environment that are influenced by fluvial and marine processes. The diverse grain size and the presence of an organic peat layer could influence the infiltration rate and so the local groundwater recharge and storage capacity. For the second study area, Heijplaat is located at the same side of the \textit{"Nieuwe Waterweg"}, but the lithology of the area is different than Rozenburg. Unlike Rozenburg, the subsurface of Heijplaat begins with a 4 meter thick anthropocentric layer that is followed by a complex, Holocene layer that consists of sandy-clay, medium-fine sand, clay, peat, and little coarse sand. An under-layer of peat indicates wet conditions in the past, which could influence the groundwater dynamics with layers of varying hydraulic properties. In Heijplaat, the peat landscape has been drained, which altered the landscape and resulted in reshaping groundwater recharge, flow conditions, and the storage ability in the area. \\
Given these facts, understanding the hydrogeological conditions of the study areas is crucial for accurately assessing groundwater dynamics. The knowledge is essential for effective groundwater management and, more specifically, setting up a framework regarding the evaluation of the level of optimization of the GWMN of Rotterdam.\\
\\
3. Does a correlation exist between network density and the degree of optimization? \\
\\
A dense network of monitoring wells does not directly indicate a high level of optimization. The research methodology that is applied for this study incorporates a reduction-optimization approach, in which a high density of monitoring wells suggests potential for optimization. Accordingly, the network density may decrease as a result of applying the reduction-optimization method, signifying improved efficiency. Ultimately, a correlation does exist between network density and the degree of optimization when looking at the information content and data quality of the monitoring wells, as a higher network density tends to offer more coverage but mostly a lower degree of optimization. Thus, while a denser network provides broader coverage of the network, optimization is achieved through strategic reduction and so enhancement of data quality, indicating a nuanced correlation between GWMN density and the efficiency of the reduction-optimization approach.\\
\\
4. How does increasing the measuring frequency and density increase the effectiveness of the groundwater monitoring network? \\
\\
A GWMN with a high quantity of data does not necessarily equate to high qualities of data. Frequent monitoring can contribute to data quality by providing more consistent and reliable datasets that are less susceptible to anomalies or data gaps. Obstacles are often shortages in data and the occurrence of data gaps. In this research study, the Python package Pastas was used, because it provides a time series modeling technique to ensure that data gaps could be filled with simulated values. Increased measuring frequency and density can reduce the reliance on simulated values, leading to a dataset that is more representative of the actual conditions of the study area. Dense and more frequent data enhances the ability to monitor changes in trends to detect potential threats or failures and execute measures promptly. Overall, enhancing the frequency and density of groundwater monitoring does not only improve data reliability and accuracy, but also improves the capacity of the network for threat detection and immediate response, elevating the effectiveness of groundwater management within City of Rotterdam. \\
\newpage
\\
5. What is the optimal monitoring reduction rate to achieve the most optimal state of the GWMN? \\
\\
\noindent
To begin with, the optimal reduction percentage differs across study areas, it is not a one-size-fits-all solution. The ideal reduction percentage, suitable for a specific neighborhood, varies depending on the characteristics of the study area. In this research study, a reduction percentage of 25\% is chosen for both study areas Rozenburg and Heijplaat for the reduction-optimization approach. A substantiation for this decision is that a minimum reduction percentage of 10\% is suitable for a reduction-extension optimization approach, as stated by Ohmer \cite{ohmer-2019} previously. When only executing a reduction-optimization approach, 25\% reduction polishes the network and the prediction accuracy of the reconstructed groundwater data is not influenced. Therefore, the remaining reduction percentages that were discussed in chapter \labelcref{chapter:RM} are not considered in the elimination process. The approach removes a number of monitoring wells, based on the reduction percentage, regarding their mutual informational relevance. Irrelevant monitoring wells are eliminated based on their informational relevance. Their groundwater data is simulated with a hydrograph reconstruction tool. The data of the eliminated network integrity is maintained through the hydrograph reconstruction tool, despite the reduction-optimization approach. Consequently, a reduction-optimization approach underpinned with a 25\% reduction, ensures the GWMN can function effectively and is able to maintain a high quality of collected data and analysis capabilities. \\
\\
The above mentioned and answered sub questions act as a tool to answer the main research question: 
\newline
\textit{"To what extent is the groundwater monitoring network of City of Rotterdam optimal?"}\\
\\
The research study encompasses two neighborhoods within City of Rotterdam to test the generated model. Therefore, it is a challenge to assess the degree of optimization of the monitoring network for the entire municipal area. The model results indicate changes in monitoring well distribution with an overall increase in the ratio of monitoring wells to surface area across both study areas, see table \labelcref{tab: concl}. The overview of the change in the ratio of monitoring well to surface area for both study areas represents the first step towards GWMNO. The methodology that is outlined in this research study offers a framework to evaluate the level of optimization. Following the framework, the approach focuses on maximizing the information content while minimizing expenses for City of Rotterdam, aligning with the Pareto Optimum principle, labelled as a reduction-optimization approach. The change in the ratio of monitoring wells to surface area, changes the efficiency of the network. By reducing the amount of monitoring wells, the expenses of the overall network are lowered as well as other parameters as frequency and intensity of required maintenance proceedings. In this research study, the model is developed and applied to Rozenburg and Heijplaat. Yet it is possible to apply the model to the remaining neighborhoods of City of Rotterdam. Interested parties have to take into account that the model is not a 'one-size-fits-all' solution, as well as a measure for the optimal density of monitoring wells in urban areas.
\begin{table}[htbp]
    \centering
     \caption{Overview of the ratio monitoring well to surface area across study areas Rozenburg and Heijplaat.}
    \begin{tabular}{|c|c|c|} \hline 
         Area&  Prior& Posterior\\ \hline 
         Rozenburg&  one well/11.17 ha& one well/15.42 ha\\ \hline 
         Heijplaat&  one well/2.78 ha& one well/3.90 ha\\ \hline
    \end{tabular}

    \label{tab: concl}
\end{table}\\
\\
The results in the table above represent the first step towards GWMNO. The methodology outlined in this research study offers a framework to evaluate the level of optimization across the entire municipal area. Following the framework, further enhancement of the municipal area within the GWMN could be pursued through an additional extension-optimization approach. This future step towards optimization is discussed in chapter \labelcref{chapter:discussion}  and  chapter \labelcref{chapter:recommendations}.

%Extending the network would be a next step towards a more optimal groundwater monitoring network, where simultaneously the informational relevance is the most important factor in determining the number and specific location of the monitoring wells. 

%A potential avenue for future optimization, to be further discussed in the chapters on discussion and recommendations, involves adopting an extended-optimization approach.





