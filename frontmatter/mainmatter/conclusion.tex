\chapter{Conclusion}
\label{chapter:conclusion}

The exploration of the groundwater monitoring network optimization within Gemeente Rotterdam has been the central point of this research study. The research addressed the following research question:\\
\\
\textit{"To what extent is the groundwater monitoring network of Gemeente Rotterdam optimal?".}
\newline
Sub questions that were introduced in the chapter \textit{"Introduction"}, are answered in the following: 

\newline
\begin{itemize}
    \item What constitutes an ‘optimal’ GWMN, particularly in terms of the ideal reduction in monitoring wells, to maintain effective monitoring while maximizing efficiency?
\end{itemize}

An optimal groundwater monitoring network can be defined as a network that achieves the maximum possible informational relevance at the lowest feasible expenses. Meaning that the lowest amount of monitoring wells ensures a high quality data through limiting construction measures regarding removal and reconstruction. An optimal GWMN with high quality data does not necessarily mean that high quantities data need to be available from the monitoring wells in the area. A network can be optimized by improving the density and measuring frequency of the monitoring wells.

\begin{itemize}
    \item How is the hydrogeological system characterized within the municipal boundaries of Rotterdam?

\end{itemize}

Overall, the hydrogeological system of Gemeente Rotterdam can be addressed as a fairly complex heterogeneous formation. Regarding the Holocene layer, the formations are marked by local gully erosion and cuts with layers that differ in structure, composition, density, and permeability. Because of the influence of the North Sea, much spatial variation in the Holocene top layer is present. To begin with Rozenburg, the neighborhood consists of a 21 meters thick layer of Holocene formations, which consist of sandy clay, medium-fine sand, peat, and coarse sand. The presence of these sediments suggest a varied deposition environment that are influenced by fluvial and marine processes. The diverse grain size and the presence of an organic peat layer could influence the infiltration rate and so the local groundwater recharge and storage capacity. The second study area Heijplaat is located at the same side of the 'Nieuwe Waterweg', the lithology of the area is different than Rozenburg. Unlike Rozenburg, the subsurface of Heijplaat starts with a 4 meter thick anthropogenic layer that is followed by a complex, Holocene layer that consists of sandy clay, mid-fine sand, clay, peat, and little coarse sand. An under layer of peat indicates wet conditions in the past, which could influence the groundwater dynamics with layers of varying hydraulic properties. In Heijplaat, the peat landscape has been drained, which altered the landscape, resulting in reshaping groundwater recharge, flow conditions, and the storage ability in the area. 

The Holocene layer is followed by a phreatic formation that is also called the Anthropogenic layer. This layer is permeable and the thickness varies between centimeters to 30 meters to the west side, close to the dunes of Hoek van Holland. In the urban areas, close to the center of Rotterdam, the top layer is strongly affected by anthropogenic influences. Like the Holocene layer, the Anthropogenic layer varies in composition, but mainly consists of sand. 





\begin{itemize}
    \item Is there a correlation between a high density of monitoring wells and a high level of optimization in the study areas?

\end{itemize}

Based on the results of the methodology, it can be concluded that a high number of monitoring wells (high density) does not necessarily mean a high level of optimization. The research methodology of the study includes a reduction method. In the context of a reduction approach, a high density of monitoring wells means that the network can be optimized. However, it is unclear if the level of optimization is improved after the eliminated monitoring wells are relocated to locations within the network with information shortages. 

\begin{itemize}
    \item Is it beneficial to optimize the measuring frequency of the data loggers and the density of the coverage area of the monitoring wells to enhance the effectiveness of the GWMN?

\end{itemize}

As was mentioned before, a GWMN with a high quantity of data does not mean that the data has a high quality. High quantities of data, especially data that is measured frequently, ensure that it is easier to execute analysis. A first obstacle in this research was namely the shortage of data and the occurrence of data gaps. Therefore, the Python package Pastas was used, because it provides a time series modeling technique to ensure that the data gaps could be filled with simulated values. Since the Pastas package also incorporates outliers in precipitation, potential evaporation, and recharge, the data set was more complete and reliable to use. 

\begin{itemize}
    \item What is the optimal reduction rate to achieve the most optimal state of the GWMN?
\end{itemize}

The optimal reduction rate could potentially differ across study areas, because the value is dependent on the number of monitoring wells as well as the data quantity of the monitoring wells in the study areas. In the current study, a reduction rate of 25\% is chosen for both Rozenburg and Heijplaat. A reduction rate of 25\% exceeds a minimum reduction rate of 10\%, but allows the network to be extended in the future. Because of the number of monitoring wells in the study areas, a network reduction of 50\% or higher seems unrealistic and unsustainable. \\
\\
In short: With a set reduction rate of 25\%, the GWMN of Rozenburg and Heijplaat could be optimized in a robust way. Monitoring wells with informational irrelevance are removed from the network, but could be reliably simulated with a hydrograph reconstruction tool. This approach that the monitoring wells do not necessarily need to be active in the system, but if data from the monitoring wells is needed, the data could be reliably simulated. The optimum reduction rate depends on more factors such as the available budget and the supply and demand of groundwater level data for local projects for example.
\newline
\\
The above mentioned and answered sub questions act as a tool to answer the main research question: 
\textit{"To what extent is the groundwater monitoring network of Gemeente Rotterdam optimal?"}\\
\\
Based on case studies of only two neighborhoods of Gemeente Rotterdam, it is difficult to conclude if the monitoring network is optimal. The GWM network's of the study areas Rozenburg and Heijplaat only take up a small part of the entire network of the municipal area of Rotterdam. Since the research methodology only focuses on reducing the current network to a more optimal state, it can be said that the network is not optimal yet. Extending the network would be a next step towards a more optimal groundwater monitoring network, where simultaneously the informational relevance is the most important factor in determining the number and specific location of the monitoring wells. 



