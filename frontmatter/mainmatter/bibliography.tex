\chapter{Bibliography}
\label{chapter:bibliography}

Aalbers, F., Voorwinden, A., & Wisse, J. (2010). Gezamenlijke aanpak grondwatermeetnet. H2O Tijdschrift, 17(1943823). https://edepot.wur.nl/340071
\newline
Centraal Bureau voor de Statistiek. (2020). Oppervlakte 2020. Onderzoek 010. https://onderzoek010.nl/jive?cat\_open\_code=cgdhegjtjcidphg2
\newline
De Vries, C. J. (1992). Stedelijk grondwaterbeheer: Het optimaliseren van het grondwatermeetnet van de gemeente Ede (No. 23). WUR: vakgroep Waterhuishouding. https://library.wur.nl/WebQuery/wurpubs/fulltext/217870
\newline
Douglas, I. (2020). Urban Hydrology. In The Routledge Handbook of Urban Ecology (2nd ed.). https://www.taylorfrancis.com/chapters/edit/10.4324/9780429506758-16/urban-hydrology-ian-douglas?context=ubx
\newline
Dufour, F. C. (1998). Hydrologische situatie en processen in West-Nederland. https://repository.tno.nl/islandora/object/uuid%3Ad6b7add5-7b50-4949-8a78-ef42b8fc549f
\newline
European Environment Agency. (2022). Europe’s groundwater — a key resource under pressure. https://www.eea.europa.eu/publications/europes-groundwater
\newline
Etikala, B., Madhav, S., & Gowd Somagouni, S. (2022). Chapter 1 - Urban Water Systems: an overview. In Urban Water Crisis and Management (Vol. 6, pp. 1–19). https://doi.org/10.1016/B978-0-323-91838-1.00016-6
\newline
Geul, K. (2022). Actualisatie Visie achtergrondmeetnet en projectmeetnetten (IB-2019-0114). Stadsontwikkeling: IBR Stedelijk Water en Geotechniek.
\newline
Gemeente Rotterdam. (n.d.). Aanpak Agniesebuurt. https://www.rotterdam.nl/aanpak-agniesebuurt
\newline
Gemeente Rotterdam. (n.d.-a). Grondwater. https://www.rotterdam.nl/grondwater
\newline
Gemeente Rotterdam. (n.d.-b). Natuurkaart. https://www.rotterdam.nl/natuurkaart
\newline
Gemeente Rotterdam. (2013). Bestemmingsplan Nieuwe dorp Heijplaat: Archeologie. In Planviewer. https://www.planviewer.nl/imro/files/NL.IMRO.0599.BP1011HeijNwDorp-oh01/t\_NL.IMRO.0599.BP1011HeijNwDorp-oh01\_4.7.html
\newline
Gemeente Rotterdam. (2020a, October). Het stedelijk watersysteem en de verantwoordelijkheden. https://www.rotterdam.nl/grp
\newline
Gemeente Rotterdam. (2020b). Van buis naar buitenruimte: Gemeentelijk Rioleringsplan Rotterdam 2021-2025. https://rotterdam.notubiz.nl/document/9422425/1/s20bb016815\_1\_39408\_tds
\newline
Gemeente Rotterdam. (2023). Het Oude Dorp. https://www.rotterdam.nl/het-oude-dorp
\newline
Hendriks, D., Passier, H., Marsman, A., Levelt, O., Lamers, N., Valstar, J., & Hu, X. (2023). Integrale Grondwaterstudie Nederland: Module 1: Landelijke analyse. In Hendriks, D., Passier, H., Marsman, a., Levelt, O., Lamers, N., Valstar, J., . . . & Hu, X. (2023). Integrale Grondwaterstudie Nederland: Module 1: Landelijke Analyse*. Deltares. https://publications.deltares.nl/11208092\_001\_0001.pdf
\newline
Hinno, R. (2021). QR decomposition. Medium. https://ristohinno.medium.com/qr-decomposition-903e8c61eaab
\newline
Hosseini, M., & Kerachian, R. (2017). A data fusion-based methodology for optimal redesign of groundwater monitoring networks. Journal of Hydrology.
\newline
Jansen, D. (2023, July 28). Saunders’ Research Onion – explained simply (With examples). Grad Coach. https://gradcoach.com/saunders-research-onion/
\newline
Janssen, J. H. M., Schelberg, J. W. C. L., & Van Rhijn, H. X. (2005). Scan stadshavens Rotterdam (Bodem en RO). In Bodem+ (No. 2004–0201). Ontwikkelingsmaatschappij Stadshavens Rotterdam N.V. https://www.bodemplus.nl/onderwerpen/bodem-ondergrond/bodembeleid/praktijkvoorbeelden/bodem-ruimtelijke/scan-stadshavens/
\newline
Jones, I. C., Shi, J., & Bradley, R. (2013). Schematic graph showing the difference between unconfined and confined aquifers. C Jones, I., Shi, J., & Bradley, R. https://www.twdb.texas.gov/groundwater/docs/GAMruns/Task13-031.pdf
\newline
Kabo, R., Baird, A. J., Bissonnette, J., Barrette, N., & Tanguay, L. (2023). Use of mixed methods in the science of hydrological extremes: What are their contributions? Hydrology, 10(6), 130. https://doi.org/10.3390/hydrology10060130
\newline
Kabo, R., Bourgault, M. A., & Bissonnette, J. F. (2022). Use of mixed methods in hydrological science: What are their contributions?https://d197for5662m48.cloudfront.net/documents/publicationstatus/103384/preprint\_pdf/122908e942299212f7603f8251e6f812.pdf
\newline
KNMI. (2024). Daggegevens van het weer in Nederland [Dataset]. https://www.knmi.nl/nederland-nu/klimatologie/daggegevens
\newline
Koomen, A. J. M., & Nieuwenhuizen, W. (2011). Klimaatbestendig Nederland: Systeemanalyse. Programmabureau Kennis voor Klimaat. https://edepot.wur.nl/307171
\newline
Massop, H. T. L., & Van der Gaast, J. W. J. (2003). Optimalisatie grondwatermeetnet Waterschap De Maaskant: Meetnet regio Oost en beschrijving grondwaterstanden periode 1990-2002 (No. 864). https://library.wur.nl/WebQuery/wurpubs/fulltext/31160
\newline
Mirzaie-Nodoushan, F., Bozorg-Haddad, O., & Loáiciga, H. A. (2017). Optimal design of groundwater-level monitoring networks. Journal of Hydroinformatics. https://www.researchgate.net/profile/Hugo-Loaiciga/publication/319063024\_Optimal\_design\_of\_groundwater-level\_monitoring\_networks/links/5aecab9baca2727bc004ecac/Optimal-design-of-groundwater-level-monitoring-networks.pdf
\newline
Ohmer, M., Liesch, T., & Goldscheider, N. (2019). On the Optimal Spatial Design for Groundwater Level Monitoring Networks. Water Resources Research, 55(11), 9454–9473. https://doi.org/10.1029/2019wr025728
\newline
Ohmer, M., Liesch, T., & Wunsch, A. (2022). Spatiotemporal optimization of groundwater monitoring networks using data-driven sparse sensing methods. In Hydrology and Earth System Sciences (Vol. 26, Issue 15, pp. 4033–4053). https://hess.copernicus.org/articles/26/4033/2022/
\newline
Ondergrondgegevens | DINOloket. (2023). TNO Geologische Dienst Nederland. https://www.dinoloket.nl/ondergrondgegevens
\newline
Ritzema, H. P., Heuvelink, G. B. M., Heinen, M., Bogaart, P. W., Van der Bolt, F. J. E., Hack-ten Broeke, M. J. D., Hoogland, T., Knotters, M., Massop, H. T. L., & Vroon, H. R. J. (2012). Meten en interpreteren van grondwaterstanden: Analyse van methodieken en nauwkeurigheid (No. 2345). Alterra. https://edepot.wur.nl/215081
\newline
RWS WVL. (2012). Handreiking juridische helderheid grondwaterbeheer. Helpdesk Water. https://www.helpdeskwater.nl/onderwerpen/wetgeving-beleid/handboek-water/thema-s/grondwater/handreiking/
\newline
Sebregts, L. (2019). Tuindorp Heijplaat vs. Tuindorp Oostzaan: Stedelijke idealen in de havens van Rotterdam en Amsterdam. https://www.leonsebregts.nl/wp-content/uploads/2019/12/Leon-Sebregts\_Tuindorp-Heijplaat-vs-Tuindorp-Oostzaan.pdf
\newline
Smith, E. C., & Swallow, S. K. (2013). Lindahl pricing for public goods and experimental auctions for the environment. In Elsevier eBooks (pp. 45–51). https://doi.org/10.1016/b978-0-12-375067-9.00107-8
\newline
Stichting Infrastructuur Kwaliteitsborging Bodembeheer [SIKB]. (2013). Plaatsen van handboringen en peilbuizen, maken van boorbeschrijvingen, nemen van grondmonsters en waterpassen: Protocol 2001. https://www.sikb.nl/doc/BRL2000/Protocol\_2001\_v3\_2\_20131212.pdf
\newline
STOWA. (1998). Methodiek voor de evaluatie en optimalisatie van routine waterkwaliteitsmeetnetten: Deel III: Stappenplan voor meetnetoptimalisatie (No. 17). https://www.stowa.nl/sites/default/files/assets/PUBLICATIES/Publicaties\%201990-2000/STOWA\%201998-17.pdf
\newline
Suringar, H. (1867). Geographical Map Rozenburg. Gemeente Atlas Van Nederland.
\newline
Suringar, H. (1867). Gemeente Charlois. Stadsarchief Rotterdam. https://stadsarchief.rotterdam.nl/zoek-en-ontdek/beeld-en-geluid/zoekresultaat-beeldgeluid/?mivast=184\&mizig=299\&miadt=184\&miview=ldt\&milang=nl\&misort=dat\%7Casc\&mistart=20\&mif4=true\&mif5=Ja\&mizk\_alle=trefwoord\%3ACharlois\%20\%28tot\%201895\%29
\newline
TNO. (1984). Ontwerp Primair Grondwaterstandsmeetnet Gelderland. Dienst Grondwaterverkenning. https://edepot.wur.nl/346574
\newline
TNO. (1996). Het voorkomen van water in de grond. In Verklarende Hydrologische Woordenlijst (Vol. 16, p. 46).
\newline
Valstar, J. R., Janssen, G. M. C. M., Marsman, A., Klein, J., & Vermooten, J. S. A. (2019). Grondwatermodel Rotterdamse havengebied: Achtergrondinformatie (No. 11201981–002). Deltares. https://publications.deltares.nl/11201981\_002.pdf
\newline
Van Bakel, P. J. T., De Braal, A. J., Geldof, G. D., Marsman, D. J., Remesal Van Merode, L. M., & Luijendijk, J. (1995). Verstedelijking en verdroging. In Nationaal Onderzoeksprogramma Verdroging (NOV rapport 4). https://www.debakelsestroom.nl/wp-content/uploads/lijst-van-publicaties-van-Jan-van-Bakel-jan-2018.pdf
\newline
Van Ganswijk, A. J., Van Claessen, F. A. M., & Veenbaas, G. (1988). Een leidraad voor de hydrologische systeembeschrijving van natuurgebieden met enkele voorbeelden. Waterbeheer Natuur Bos en Landschap. https://edepot.wur.nl/192351
\newline
Van ’t Zelfde, A. (2011). De waarde van een gemeentelijk grondwatermeetnet. H2O Tijdschrift, 44(17), 11970316. https://edepot.wur.nl/339546
\newline
Witte, J. P. M., Aggenbach, C. J. S., & Runhaar, J. (2008). Grondwater voor natuur: deel II (No. 607300003). KIWA Water Research Nieuwegein. https://rivm.openrepository.com/bitstream/handle/10029/261731/607300003.pdf?sequence=3&isAllowed=y

