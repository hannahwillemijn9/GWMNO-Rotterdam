\chapter{Introduction}
\label{chapter:introduction}

%\emph{... \cite{example-article}}

\section{Background}

The municipality of Rotterdam monitors an extensive groundwater monitoring network, consisting of approximately 2000 groundwater wells. The wells cover the entire municipal area of Rotterdam, reaching from urban to rural locations. The groundwater level in the monitoring wells gives insight into the groundwater level in the city districts, but is not intended to be used for monitoring on household level. Measuring groundwater is important to monitor fluctuations in groundwater levels in specific areas of attention within the municipality. Groundwater levels that are too high or too low are likely to result in problems on the topic of foundations and settlements, (ground)water nuisance in basements. The observed data by the municipality is used for civil technical measures and asset management. Wells that are constructed in public places are the city's responsibility. Because of the municipal duty to take care of the monitoring wells, it is of importance to monitor the groundwater level, but also to investigate whether the network density and the location of the monitoring wells are efficient enough to gain optimal results.


\section{Problem statement}

The degree of well coverage and monitoring methods differ between the city districts. One of the reasons is that the City of Rotterdam only investigates the coverage of a subarea when a well is expired; they determine the necessity of well replacement and if the new well can be placed on a more optimal and efficient location. Consequently, the number of groundwater wells has been stagnant throughout the years and it is not known if the current number of monitoring wells is the most suitable number to retrieve additional groundwater information \cite{geul-2022}. Since groundwater is an integral part of the hydrological cycle, it is key to monitor and anticipate the groundwater levels. Therefore, it is important to research the design of the network, analyze whether the network is optimal, and if monitoring wells need to be removed or reconstructed in a later stage \cite{european-environment-agency-2022}. An insufficient groundwater monitoring network could lead to consequences for urban planning, infrastructure, and environmental pollution. An insufficient network might form an obstacle regarding Rotterdam’s ability to mitigate and adapt to climate change impacts such as increased precipitation variability. 
\newpage
\section{Research objective and questions}

Engaging the outlined research questions into a comprehensive approach, incorporating theoretical and empirical analysis. Even though the number of monitoring wells has been stagnant over the past years, the network is currently an actively monitored groundwater network. However, it is not known if the network is ideal yet. Therefore, the objective of the research study is to investigate the degree of optimization of the groundwater monitoring network of Rotterdam. \\
\\
The main research question is as follows:\\
\textbf{To what extent is the groundwater monitoring network of the municipality of Rotterdam optimal?}\\
\\
With supporting subquestions to answer the main research question:\\
\\
Defining optimal\\
What constitutes an ‘optimal’ groundwater monitoring network, particularly in terms of the ideal reduction in monitoring wells, to maintain effective monitoring while maximizing efficiency?\\
\\
Characterizing the hydrogeological system of Rotterdam\\
To effectively optimize the groundwater monitoring network, the hydrogeological and geographical system of Rotterdam have to be considered. Geological formations are mapped and hydrological data is analyzed to answer the question: 
\\
How is the hydrogeological system characterized within the municipal boundaries of Rotterdam?\\
\\
Correlation between network density and optimization\\ 
Researching the correlation between the density of the monitoring wells and the degree of optimization incorporates a time series analysis and QR factorization. These analysis are done to answer the question:
\\
Is there a correlation between a high density of monitoring wells and a high level of optimization in the study areas?\\
\\
Benefits of optimizing measuring frequency and coverage area\\
Is it beneficial to optimize the measuring frequency of the data loggers and the density of the coverage area of the monitoring wells to enhance the effectiveness of the GWMN?\\
\\
Reduction rates for groundwater monitoring network optimization\\
Identifying the best reduction rate to achieve the most optimal performance of the GWMN.The impact of varying amounts of monitoring wells, well distribution, and data availability influence the most optimal reduction rate for network reduction. Comparing these factors with each other across multiple neighborhoods will form an answer to the question: \\
What is the optimal reduction rate to achieve the most optimal state of the GWMN?

\section{Social and scientific relevance}

This study holds significant relevance as it delves into the aspects of groundwater monitoring networks within the municipal area of Rotterdam, particularly within urban and industrial settings. The primary objective is to assess the level of optimization of the groundwater monitoring network in Rotterdam through the development of a generic QR factorization model. The outcomes of the research have the potential to serve as a valuable point of reference for other urban areas that face similar groundwater challenges. Accordingly, it is imperative to examine the GWMN to determine that sufficient and necessary groundwater data are collected to make informed decisions for integrated groundwater management. The examination should include a critical review on the spatial distribution and network density of the GWMN as well as monitoring frequency. Overall, the study aspires to develop an innovative approach to monitor and enhance the optimization of Rotterdam’s GWMN, contributing to social and scientific advancements. 

\section{Reading guide}

In the chapter "Literature Review" multiple research methodologies are analyzed and compared in order to criticize whether the methodology could be applied to the current research study. At the end of the chapter, a short summary is written to determine which methodology is suitable. Moving on to the "Theoretical Background", where context is shared regarding the geological and hydrological context of the research area. As well as the current management and policy standards that are followed on local and national scale. Starting with the "Research Methodology", the chapter starts with a short overview of the philosophy and approach that is considered. From there, a description of the research design is given, continuing with the subsections "Data collection and preparation", where the programs qGIS, PROWAT, and Pastas time series modeling are introduced and followed by the data analysis section. Both of the case studies, the neighborhoods Rozenburg and Heijplaat, are included in these chapters. Followed by the section "QR Factorization", which is the actual optimization study. These subsections are incorporated in the two case studies of Rozenburg and Heijplaat. 

To write: "Results”, “Discussion”, “Conclusion” reading guide. 





