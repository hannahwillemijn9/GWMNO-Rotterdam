\chapter{Discussion}                                                                                 
\label{chapter:discussion}
The discussion critically examines the findings on optimizing the groundwater monitoring network of Rotterdam, based on groundwater level data from the period of 2020-2024. The discussion aims to give insight into the implications and suggested directions for future investigation and application of the model into the field. \\
\newline
\noindent
\textbf{Observed data}\\
The substantiation for the time period of the observed data, ranging from 2010-2024, to a more narrow scope of 2020-2024 potentially affects the precision and reliability of the results of the stress model. More specifically, a wider time frame may provide a better view of the groundwater dynamics and anomalies, referring to the outliers recorded in the observed manual measurements between 2014 and 2018. Nevertheless, an extended time period might introduce variability, which does not improve model performance. A shorter time frame, however, might not capture the entire range of hydrological events that are necessary to create a robust model. On the other hand, a wider time frame can positively influence the results of the model, but taking into account the effects of external factors such as spatial planning or inefficient sewage systems, could influence variability of the data. The observed data is collected by field service employees of the field measurement service of City of Rotterdam through manually collected measurements. The measurements take place at the end of the month, so sampling every month. The manual collection method could be the reason as to why monitoring wells with the manually collected data have more data inconsistencies. Besides that, the transition to different field service employees in the year 2014 and 2018 possibly resulted in differences, and maybe even errors, in the approach of data collection. From 2019 on, groundwater data is collected solely by City of Rotterdam. \\
\\
\textbf{Data preparation}\\
Since a combination of data loggers and a manual collection method is applied for multiple neighborhoods within the municipality, data gaps are identified. Time series modeling appeared to be an effective approach for bridging data gaps with dependable data. A query is, however, if the measuring frequency of the City of Rotterdam should be adjusted, to hourly measurements for example, in order to produce more frequent data. This way, data gaps do not have to be bridged with the use of Pastas time series modeling. Pastas time series modeling is based on simulation data of precipitation [m/day] and potential evaporation [m/day] data, in order to calculate the recharge rate [m/day]. For this research study, the data is based on weather station 344 in Rotterdam, The Netherlands. Weather station 344 is the station that is commonly used to retrieve data for City of Rotterdam. However, as the crow flies, both Rozenburg and Heijplaat have a distance of approximately 15 kilometers from the weather station. Therefore, the measured data at weather station 344 can differ between the in reality observed and measured precipitation and evaporation in the neighborhoods Rozenburg and Heijplaat. However, the use of Pastas time series modeling presents a limitation because of its focus on evaporation and precipitation parameters. Hydrological parameters such as hydraulic conductivity and drainage are not considered in the model. This raises questions regarding data gaps: 1) How should data gaps be addressed upon identification?; 2) How does inclusion of simulated data, to fill data gaps, influence the model results? Addressing these queries in future research should be a priority to optimize the model and its results, see chapter \labelcref{chapter:recommendations}. \\
\\
\textbf{QR factorization}\\
The process of QR factorization and the additional reduction of monitoring wells introduces the relationship between network optimization and the RMSE performance metric. The findings of the model suggest that an optimized network is marked by both a reduced number of monitoring wells and a lower RMSE for the network, indicating that the level of optimization is correlated to the quality of data derived from the chosen monitoring locations rather than the quantity of data. Therefore, achieving a high level of optimization necessitates a careful selection of monitoring wells that ensures data quality for the monitoring network. Data quality means, in this context, the possibility to lower the RMSE without influencing the data quantity. Strategic selection of monitoring wells through QR factorization enables the network to maintain and improve the model performance with fewer monitoring wells. \\
\\
\textbf{Sensitivity and performance metrics}\\
It is necessary to question whether the RMSE and MAE are the only sensitive parameters that reflect model accuracy. The RMSE describes a measure of the magnitude of errors and it remains unaffected by changes in the variability of the input of the observed groundwater level data. As stated by Ohmer, a greater RMSE, in comparison with a MAE, indicates a larger variation of errors and highlights areas of variability in the performance of the reconstruction of the eliminated monitoring wells \cite{ohmer-2019}. It has been observed that monitoring wells with a lower variability in data are likely to add less informational relevance to the network and are, thus, more likely to be eliminated from the network. Besides, the added value of centering the input data is questionable. Centering is an often used method for preprocessing data, when the reference point is the mean of the dataset. The result of centralization encompasses an analysis of class-centered data that examines the variance within sample groups, but ignores variation in data within the sample groups. Applying global and local centering, followed by decentralizing the data in the reconstruction phase, influences the outcomes of the RMSE, MAE, $R^2$, and rBIAS of the monitoring wells. Essentially, applying a centralization approach ensures that the mean of the observed data is as near as zero as achievable. \\
\\
\textbf{Implications of reduction rates}\\
In the research methodology, reduction percentages ranging from 10-90\% are tested. When the reduction percentage reaches 25\%, monitoring wells with the lowest information content are identified and eliminated from the network. With a reduction percentage of 25\%, the network is polished, the prediction accuracy is slightly influenced, and there is the least amount of network destruction. Therefore, the remaining reduction percentages of 10, 50, 75, and 90\% are not considered in the elimination process. The elimination process is based on 25\% for both study areas Rozenburg and Heijplaat. It is hypothesized that the results according to a reduction of 25\% have a lower loss of prediction accuracy for the RMSE, reconstruction accuracy and performance metrics (RMSE, MAE, $R^2$, rBIAS) of the eliminated monitoring wells in comparison with the other stated reduction scenarios. A low prediction accuracy, meaning low performance metrics, could be the result of previous processes like the type of data collection and modeling implications of Pastas time series modeling.\\
\\
\textbf{Model robustness and variability}\\
The RMSE remains unchanged when increasing or decreasing the variability of the input data. This suggests that the model performance is consistent and the robustness to changes in variability is relatively high. Determining the robustness is necessary to examine whether the model is able to remain consistent when experiencing a variability of hydrological conditions and input scenarios. Accordingly, the elimination of monitoring wells and the choice of study period requires an equilibrium regarding data quality and quantity. Both the RMSE and MAE are performance metrics that evaluate model performance, because they ensure perspective in model sensitivity and performance. From the model results, it has become clear that the RMSE is not sensitive to changes in variability input. At the same time, the direct influence of the MAE on the results marks the complexity of GWMNO. 

Additionally, both results of Rozenburg and Heijplaat mark a significant difference between the observed and reconstructed groundwater level data of the eliminated monitoring wells. Even though the performance metrics of the eliminated monitoring wells mirror a good fit of the reconstructed groundwater level data with the observed groundwater level data by City of Rotterdam. An explanation for the difference between the results of the performance metrics and Welch's t-Test could be the objective of the assessment, meaning that the Welch's t-Test indicates the probability of observed data if the hypothesis is true. Nonetheless, the performance metrics can address a good fit of the reconstructed and observed data, a t-Test can address if there are statistically significant differences between the two data sets. The combination of assessing both model performance and a statistical test provides insight in model accuracy that could be beneficial for further model development and application. \\
\\
\noindent
\textbf{Implementation of the model in the field}\\
Before the research study started, the objective was set to create a generic model to be able to test the level of optimization for the entire municipal area of Rotterdam on neighborhood scale. Two case study areas were selected: Rozenburg and Heijplaat, to test and train the model with a low number of monitoring wells. A query is, however, if the methodology and model would be robust enough to handle external factors in urban areas, that correlate to the location of wells, that would possibly have an influence on groundwater levels locally. For instance: The design of the filter of the well in correlation with clay or sand, the proximity to water bodies, and construction activities that might affect surface runoff or impact the sewage system. The reduction-optimization approach relies on the proposition that the physical system remains unchanged in its present state in order to operate effectively. 






