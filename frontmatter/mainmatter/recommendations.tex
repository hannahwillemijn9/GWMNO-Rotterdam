\chapter{Recommendations}
\label{chapter:recommendations}

Based on the results of the research study, the main suggestion for this research study is to execute additional research regarding a reduction-extension approach.\\
\\

\textbf{1D reduction and 2D extension methodology}\\
\\
Instead of researching 1D hydrograph data only, it is also possible to engage more into groundwater level contours to ensure reconstruction and extension are included. Through this approach, an indication of the predicted expenses can be depicted, which could persuade other municipalities or interested parties to optimize their local groundwater monitoring network. The gradual cost function can be used where the weighting increases with distance to an infrastructural object or similar. In the 2D method, weighting depends on the system, basis, and cost function and should be adjusted to the specific case study. \\
\\
\textbf{Municipal area of Rotterdam}\\
\\
The boundaries of the municipal area of Rotterdam reach from Hoek van Holland to Kralingen and the industrial areas. Gemeente Rotterdam desires to construct additional monitoring wells in public spaces throughout the industrial areas. Use of a reduction-extension model could be useful to examine if they should be removed, replaced or adding monitoring wells is of additional value to the network. And, if the network reaches its optimal state. \\
\\
\textbf{Optimizing the model}\\
\\
A research limitation includes the unknown mathematical calculations that are present in the libraries of the used Python packages, for example PySensors. The calculations that are present in the libraries of the PySensor package defines the mathematical substantiation for the reconstruction of the groundwater level data. Therefore, it is recommended to determine which formulas and calculations have been carried out to achieve the results from the model. Potentially, these formulas could be used to forecast and reconstruct groundwater level data in hydrographs.

