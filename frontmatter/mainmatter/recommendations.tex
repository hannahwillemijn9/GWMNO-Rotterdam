\chapter{Recommendations}
\label{chapter:recommendations}

Based on the results of the research study, the main suggestion for this study is to execute additional research regarding a reduction-extension approach.\\
\\
\textbf{1D reduction and 2D extension methodology}\\
Instead of researching 1D hydrograph data only, it is possible to engage more into groundwater level contours to ensure reconstruction and extension are included. Through this approach, an indication of the predicted expenses can be depicted, which could persuade municipalities or interested parties to optimize their local groundwater monitoring network. The gradual cost function is applicable in scenarios where weighting increases with the proximity to an infrastructural object. With the 2D approach, the weighting depends upon the system, basis, and cost function and should be adjusted to suit the research area. Additionally, Ohmer states that the reconstruction error for the reduction-extension optimization approach is slightly lower than the results of the reduction-optimization approach \cite{ohmer-2019}. \\
\\
\textbf{Municipal area of Rotterdam}\\
The boundaries of the municipal area of Rotterdam reach from Hoek van Holland to Kralingen and the industrial areas. Gemeente Rotterdam desires to construct additional monitoring wells in public spaces throughout the industrial areas. Use of a reduction-extension model could be useful to examine if they should be removed, replaced or extended by adding monitoring wells to the network. And, to determine if the network reaches its optimal state. \\
\\
\textbf{Optimizing the model}\\
The optimization-reduction approach could be refined by expanding the hydrological perspective of the model. In the current study, the hydrological parameters evaporation and precipitation are included in Pastas time series modeling. However, this gives the time series a limited, hydrological perspective, as mentioned in \labelcref{chapter:discussion}. The time series model could be extended in the future with parameters as hydraulic conductivity or drainage for example. Another research limitation regarding the model includes the unknown mathematical calculations that are present in the libraries of the used Python packages, for example PySensors. The calculations that are present in the libraries of  the PySensors package defines the mathematical substantiation for the reconstruction of the groundwater level data. Therefore, it is recommended to determine which formulas and calculations have been carried out to achieve the results from the model. Potentially, these formulas could be used to forecast and reconstruct groundwater level data in hydrographs. 



