\chapter{Research Methodology}
\label{chapter:RM}
%\emph{The research methodology describes a chronological order of methodologies used to execute research. Components are research philosophy and approach, research design, data collection and preparation, data-driven sparse-sensing techniques and QR-based algorithm, and a case study in which two study areas are discussed.}

%\section{Research philosophy and approach}
%Within this research study, a positivism research philosophy is adopted. Positivism can be illustrated as a scientific conception to which a phenomenon can be known through empirical data based on natural processes. The concept relies on quantitative measurements and methods to illustrate the hydrological phenomenon and predict it in this way. As stated by Kabo et al. (2023), a mathematical model of hydrological flow is based on a positivist paradigm through which the natural world can be explained. According to Jansen (2023), a research methodology with a positivist perspective can only acquire data through measurements and observations. Data used in this research study is acquired through manual and sensor measurements of monitoring wells inherited by City of Rotterdam. Next to the research philosophy, the research approach that is planned to be adopted is a quantitative research method. This research method involves the collection and analysis of numerical data to test hypotheses, examine patterns, and establish relationships between variables. The objective is to quantify and generalize findings. A quantitative research method uses a deductive approach, meaning a clear theory or hypothesis has to be researched. Consequently, the theory or hypothesis is tested, using specific observations or data. The research approach uses a top-down approach, meaning the theory or hypothesis that is available is first assessed instead of assessing data or observations (Jansen, 2023).

\section{Research design}
The research design outlines the framework guiding the research, aligned with the objectives. The study aims at developing a generic model on neighborhood scale and intends to assess the optimization level within Rotterdam's GWMN. The methodology is structured into a dual-part research design: 

\subsection{Descriptive design}
The initial phase aims to characterize the present state of the system, utilizing hydrogeological and geographical data of the research areas. The system description involves analyzing the distribution and coverage of monitoring wells within the municipality without manipulating variables. The descriptive design develops a detailed visualization of the existing GWMN, including parameters as distribution, coverage density, and groundwater level measurements.

\subsection{Quasi-experimental design}
The second phase includes an approach that manipulates the GWMN by simulating data and removing monitoring wells. These actions have the purpose to observe changes regarding network optimization. Initially, the research study includes 14 city districts and 4 industrial areas, however, a generic model is set-up to assess the optimization on neighborhood scale. Only two specific research areas are included in the chapters \labelcref{chapter:RM}  and \labelcref{chapter:results}. The choice of the two specific research areas is based on factors such as data collection type, monitoring frequency, data availability, and density of the coverage area. Data loggers ensure that inaccessible and vulnerable locations within the city districts still can be frequently monitored through this approach. Certain monitoring wells are on inaccessible locations within the municipal area, however, through the use of data loggers these locations can still be actively monitored. The design continues with a quantitative method to analyze the impact of changes regarding the reduction-optimization method over a longitudinal period.

\section{Data collection and preparation}
Within the research study, the time period includes 01-01-2020 to 01-01-2024. From 2010 to 2014, the groundwater data was collected by City of Rotterdam. In the period of 2014-2018, however, the field measurements were carried out by an external contracting company. The field measurements take place at the end of every month, executing a measurement every month. Since 2019, groundwater data is again collected by City of Rotterdam, meaning that multiple monitoring wells throughout the municipal area are transferred to data loggers instead of the manual collection method. Because of this transition, inaccessible locations were still monitored and data is collected more frequently; every 4 hours instead of every month.

\subsection{QGIS and PROWAT}
Starting with the first step of the data collection, the open-source QGIS and Sensor Management application of City of Rotterdam allows displaying monitoring data for all unique monitoring wells within the municipal area. In QGIS, study areas of preference can be selected and transferred to the PROWAT system. PROWAT allows the data to be converted to an Excel data file. After which, data can be imported in the programming language Python.

\subsection{Pastas time series modeling}
The quasi-experimental research design starts with Pastas, a Python package made to analyze a sequence of data points that are collected over specific interval of time \cite{collenteur-2019}. In the original dataset, data gaps are present, resulting in potential problems later on in the development of the model. At first, a basic model is generated to visualize a simple time series model to simulate groundwater levels in the research areas.

\subsubsection{Dependent time series}
A time series of groundwater levels is imported through the Pastas package. The dependent time series data describe the observed time series monitored by City of Rotterdam. The results of the dependent time series describe the groundwater level [m NAP] over a period of 01-01-2010 to 01-01-2024. Unique monitoring wells of the neighborhoods Rozenburg and Heijplaat are visualized. The purpose is to depict figures that visually represent seasonal changes and variabilities as well. 

\subsubsection{Independent time series}
The process includes incorporating an independent time series from an external database. The integration is being done for both precipitation and potential evaporation data. By utilizing both datasets, it is possible to estimate the recharge to the groundwater [m/day]. An additional stress model can be created through the calculation of the evaporation factor, the actual evaporation as a factor. The recharge is created with a linear recharge model, a combination of precipitation and evaporation series and so a parameter for the evaporation factor. The data is sourced from KNMI at weather station 344 in Rotterdam, The Netherlands. The depicted figures illustrate the recharge rate [m/day] spanning from 01-01-2010 to 01-01-2024, as well as the individual time series for precipitation [m/day] and potential evaporation [m/day].

\subsubsection{Time series model}
Once the dependent time series (observed GWL data) and independent time series (precipitation and potential evaporation data) are developed and plotted, the actual time series model can be created. The dependent and independent time series are imported to the actual time series model. The imported time series are verified for data inconsistencies and missing values. The model is improved through assessing optimal model parameters. The solution of the calculation can be plotted in a time series plot, which describes the observed data as a scatter plot and the simulation data, according to the independent data in a line plot. The actual time series model includes "Stress Model 1". The depicted data translates the groundwater level [m NAP] over a time period of 01-01-2010 to 01-01-2024. Besides the time series model, it is possible to accommodate an overview of the model. In the overview plot, information regarding the residuals, noise, recharge, and stresses are displayed. A full description of "Stress Model 1" is stated in Appendix \labelcref{appendix: A}.

\subsubsection{Statistical substantiation}
In the end, a summary of the developed performance metrics is provided. The performance of the models that are created previously, is visualized in a bar plot with performance metrics of RMSE, $R^2$, and EVP. Nevertheless, based on statistical data only, it can not be provided if a model performs good or bad. Therefore, an additional statistical test is executed. The Welch's t-Test compares data groups: data logger and manual collection method without assuming equal data variances \cite{ahad-2014}. The t-Test is carried out to criticize whether a dataset with a combination of measurement types: data logger, manual collection or a combination of these two is recommended to be used further in the research study. A Welch's t-Test is a suitable approach, because the denominator provides the possibility that two data groups occur to have unequal variances. The t-Test is determined by the mean of the two data groups, the variances of both data groups, and the sample sizes of the two data groups. Furthermore, the Welch's t-Test suits, since a difference occurs between the data quantity of the data loggers and manual collection method. Nonetheless, in the context of environmental sciences, a perspective beyond statistics is key: Other factors such as the reliability and quality of the data and preference of stakeholders can shape the final decision regarding the overall substantiation of the data group that might be included in the research study. In summary, adopting a comprehensive approach that goes beyond statistical significance is crucial. The decision has been taken to proceed with the research using the data logger group. The observed and simulated data logger measurements are combined into a new data frame, with which the research study will be proceeded with.

\section{QR factorization}
A data-driven algorithm, based on the Python package PySensors, is designed to optimize groundwater monitoring networks by determining the optimal number and location of monitoring wells for temporal and spatial level reconstruction. The algorithm can be explained by the dependency of data in order to make a certain decision, prediction or optimization process. Existing data from City of Rotterdam is used to study patterns, trends, and relationships between parameters. The method was first obtained by Ohmer at the KIT and applied the methodology to a study area in Germany \cite{ohmer-2022}. The approach is included in the optimization approach regarding City of Rotterdam. The network reduction-optimization method is utilized on the GWMN, using one-dimensional hydrograph data. The main process of the algorithm is the use of QR factorization, a mathematical approach that breaks down matrix \(A\) into the product of two matrices: \(Q\) and \(R\). \(Q\) is an orthogonal matrix that transforms the original dataset into a new dataset with independent columns. \(Q\) is essential for evaluating the unique contribution of each monitoring well. \(R\) is a triangular matrix that ranks the monitoring wells based on their informational relevance. In the algorithm, matrix \(A\) encompasses the dataset that is created through Pastas time series modeling. In the dataset, every row corresponds to a different observation and every column represents a unique monitoring well. Appendix \labelcref{appendix: B} describes a more in-depth mathematical background of the QR factorization approach. \\
\\
The optimization approach enables the network to be reduced along the Pareto front, balancing information loss against cost savings. To achieve the most accurate reconstruction with the least number of monitoring wells, the algorithm tests various reduction scenarios (10-25-50-75-90\%) to identify the lowest reconstruction error for the monitoring network and so the optimal number of monitoring wells. The number of monitoring wells is indicated as a so-called threshold value, explaining the trade of between the maximum information value of the monitoring wells and the minimum costs, in which the maximum information content is characterized by the reconstruction error and the minimum cost is equivalent to the number of monitoring wells \cite{wu-2004}. It can, however, be possible that the optimal reduction percentage differs between the study areas. A reduction percentage of at least 25\% should be applied to describe the system dynamics in the study areas, while a reduction percentage exceeding 75 percent would influence the accuracy of the reduction results \cite{ohmer-2022}. \\
\\
In \href{https://github.com/hannahwillemijn9/GWMNO-Rotterdam.git}{Github}, a detailed script regarding the QR factorization method in the model is described. The Python package "PySensors" contains a data-driven genetic algorithm for sparse sensor selection and signal reconstruction with dimensional reduction, see appendix \labelcref{appendix: B}.

\section{Data analysis}
A quasi-experimental design is employed to critically examine, analysis, and compare data, derived from data loggers and the manual collection method by City of Rotterdam. Firstly, the dataset undergoes a preprocessing phase in which a division is made into a training and test dataset, designed to test the predictiveness and performance of the model. Continuing, a QR factorization approach is applied and offers insight into the impact of different data collection strategies on model performance. Based on the informational relevance of the monitoring wells within the study area, a hierarchical list can be generated. The QR factorization approach applies a quasi-experimental evaluation of the impact of well reduction on the overall model performance. A systematic reduction of the network enables to assess the correlation between groundwater monitoring network reduction and the accuracy of the model. Assessing the reduction scenarios ensures to gain understanding on how different data collection methods affect the robustness and accuracy of groundwater level predictions. After a reduction scenario is chosen, the scenario eliminates groundwater monitoring wells with less informational relevance from the network. The eliminated monitoring wells are visualized in hydrographs and performance metrics, in order to substantiate and compare the performance of the observed and reconstructed data. 











