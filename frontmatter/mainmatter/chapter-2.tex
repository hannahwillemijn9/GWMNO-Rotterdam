\chapter{Literature Review}
\label{chapter:Literature Review}
%\textit{The chapter "Literature Review" reviews existing literature that might be relevant to the objective. The chapter demonstrates an understanding in the field of groundwater, identifies knowledge gaps in current research studies, and positions the research study within a broader academic context.}

The importance of monitoring groundwater quantities and collecting comprehensive data on the spatiotemporal dynamics of aquifers is highlighted. Understanding the movement and characteristics of groundwater over time allows the development of sustainable strategies \cite{ohmer-2019}. A perception is the design of an optimal groundwater monitoring network. The ideal network achieves the maximum possible information content at the lowest feasible expenses. The perception is in close alignment with the Pareto Optimum, which can be described as a standard that illustrates a situation without any further improvements of society's well-being \cite{smith-2013}. The optimum can be made through a re-allocation of resources that makes at least one person better-off without making someone else worse-off \cite{smith-2013}.  In the context of groundwater management, the principle suggests creating a system that maximizes efficiency and benefits without incurring additional costs or disadvantages to any of the concerning stakeholders. As written by STOWA, optimizing the density of monitoring wells in a groundwater monitoring network is a process that requires a clear definition of the term 'optimization' \cite{stowa-1998}. STOWA provides a comprehensive framework, outlining a six-step methodology in order to determine the degree of optimization in monitoring networks \cite{stowa-1998}.

\begin{enumerate}
    \item The network is optimal when it is able to continuously provide accurate data on the status of the groundwater. 
\textit{If that is true,} the use of spatiotemporal models as process modeling and geostatistical modeling are recommended. \textbf{
}    \item The network is optimal when it is able to translate data into accessible information.
\textit{If that is true,} mapping of historical, current, and future data are recommended.
    \item The network is optimal when every monitoring well is representative for the water quantity and quality in the area or for a specific water body. 
\textit{If that is true,} the representativeness of the research area has to be assessed through project-based studies.\textbf{
}    \item The network is optimal when one or more area covering images of the water quantity and quality can be obtained. 
\textit{If that is true,} based on geographical information and system knowledge it can be researched to what extent the measuring points are researched throughout the research area. The use of a Geographic Information System (GIS) is required. \textbf{
}    \item The network is optimal when every monitoring well provides unique information without overlapping neighboring monitoring wells. 
\textit{If that is true,} based on statistical tests it can be researched to what extent the monitoring wells show overlapping in data. 
\item The network reaches an optimum when the above-mentioned steps are effectively combined and aligned with the other monitoring objectives.
\\

\end{enumerate}
The development of monitoring networks has been an occasional process that is shaped by varying demands and interests, specific to the time of their construction. Monitoring networks grow historically based on ad-hoc installation of monitoring wells. Installed monitoring wells have acquired the right to exist, but all have their own advantages and disadvantages. Practically, a combination of the previously mentioned six-step methodology is required \cite{stowa-1998}. It is argued that refining the frequency of measurements at a monitoring well is ineffectual if the monitoring wells lack a rational for its existence. Certain monitoring wells may prove to be suitable for trend analysis, because of its statistical characteristics. Adjustments in measuring density must consider a range of monitoring objectives beyond optimization.\\
\\
From a research perspective, Mirzaie-Nodoushan proposed paired objectives for the establishment of a groundwater monitoring network \cite{mirzaie-nodoushan-2017}. Firstly, to minimize the number of monitoring wells, and secondly, to minimize the root mean square error (RMSE) between observed and simulated data in the aquifer under study. The latter objective focuses on minimizing the average of the squares of all prediction errors. \\
In a study by former waterboard \textit{"Maaskant"}, presently known as waterboard Aa en Maas, the intention was to complement the regional water system, focusing on the unconfined groundwater monitoring network \cite{massop-2003}. In this study, Massop and van der Gaast detail the distribution of the network within the study area and evaluate it against specific criteria. The criteria include the representativeness of the GWMN in relation to the spatial characteristics of the research area and and its capability to distinguish between different clusters. The study represents the network into clusters based on observable characteristics, assigning average values for hydrotype, groundwater depth classes, soil type, land use, water supply, and distance ranking. Logically, clusters with analogous characteristics are grouped. For each cluster, key hydrological parameters such as average groundwater level [\(h_{\text{average}}\)], system inertia [\(\delta\)], and recharge sensitivity [\(\omega\)] are calculated, which are influenced by drainage resistance and the storage coefficient, affecting groundwater level reactions to precipitation and surface water level changes.  More specifically, the system inertia helps to create an understanding of the relationship between precipitation and the influence on groundwater recharge, while the recharge sensitivity describes the sensitivity of the groundwater system to changes through external influences as precipitation or local pumping activities. The minimum and maximum values for these three parameters are estimated, as well as the median values (50\% higher and 50\% lower) for every cluster of monitoring wells in the study area. The median value is chosen for the estimation, because the median is less sensitive to outliers in data. The most representative location for each cluster of monitoring wells is determined by the smallest deviation for the cluster \cite{massop-2003}. \\
\\
Reproducing the approach by Massop and van der Gaast \cite{massop-2003}, Hosseini and Kerachian argue that the purpose of groundwater monitoring is to deepen the understanding of the hydrogeological system through continuous data collection of the groundwater quantity and/or quality \cite{hosseini-2017}. The approach divides the research area into clusters and employs hexagonal gridding with use of the Thiessen polygon approach in order to examine spatial sampling patterns. They integrate three criteria to set a base for priority areas for monitoring, based on aquifer traits and groundwater level fluctuations. The approach combines three criteria to identify priority areas for monitoring which are based on aquifer characteristics and groundwater fluctuations (marginal entropy of water table levels; estimation error variances of mean values of water table levels; estimation values of long-term changes in groundwater level). Spatiotemporal kriging and ANN are used for predicting groundwater level variations. Data fusion techniques are deployed to refine prediction precision, reducing the RMSE. The Value Of Information technique is instrumental in identifying monitoring sites that yield maximum informational value within each priority area and suggests varying sampling frequencies. The approach by Hosseini and Kerachian presents a cost-effective strategy for redesigning a GWMN \cite{hosseini-2017}.\\
\\
Overall, the literature on GWMN accentuates a transition from historical development towards optimized configurations. The literature by STOWA \cite{stowa-1998} and Hosseini and Kerachian \cite{hosseini-2017} reflect consensus on the need for the interpretation of monitoring wells and their data collection in terms of frequency and data utility. While Mirzaie-Nodoushan focuses on minimizing both the number of monitoring wells and the prediction errors \cite{mirzaie-nodoushan-2017}, Massop and van der Gaast prioritize the representativeness and clustering of monitoring wells to visualize the heterogeneity of the study area \cite{massop-2003}. Insights by Ohmer emphasize the role of monitoring in groundwater management that is in line with the Pareto Optimum \cite{ohmer-2019}, to achieve the maximum efficiency of the GWMN \cite{smith-2013}. There is a clear perspective for the future of groundwater monitoring: Networks that are not only cost-effective and efficient, but also robust in terms of their capacity to provide high quality data. The literature review provides a foundation for the development of a generic model that evaluates the performance of the GWMN on neighborhood scale and is proficient in addressing and handling challenges in groundwater management. 











